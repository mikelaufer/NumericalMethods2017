% Created 2018-01-01 Mon 22:33
% Intended LaTeX compiler: pdflatex
\documentclass[11pt]{article}
\usepackage[utf8]{inputenc}
\usepackage[T1]{fontenc}
\usepackage{graphicx}
\usepackage{grffile}
\usepackage{longtable}
\usepackage{wrapfig}
\usepackage{rotating}
\usepackage[normalem]{ulem}
\usepackage{amsmath}
\usepackage{textcomp}
\usepackage{amssymb}
\usepackage{capt-of}
\usepackage{hyperref}
\usepackage{minted}
\author{mike}
\date{\today}
\title{}
\hypersetup{
 pdfauthor={mike},
 pdftitle={},
 pdfkeywords={},
 pdfsubject={},
 pdfcreator={Emacs 25.3.1 (Org mode 9.1.4)}, 
 pdflang={English}}
\begin{document}

\tableofcontents

\section{7.1 Gauss-divergence theorem}
\label{sec:org577de4b}
Area (volume in 2D) is given by equation 7.40:
\[
V_{0}=\frac{1}{2} \left( \sum_{1}^{N_{f,O}} n_{x,f}x_{f}A_{f} + \sum_{1}^{N_{f,O}} n_{y,f}y_{f}A_{f} \right)
\]
and in 3D:
\[
V_{0}=\frac{1}{3} \left( \sum_{1}^{N_{f,O}} n_{x,f}x_{f}A_{f} + \sum_{1}^{N_{f,O}} n_{y,f}y_{f}A_{f} + \sum_{1}^{N_{f,O}} n_{z,f}z_{f}A_{f} \right)
\]


The face normals in 2D, n\(_{\text{x,f}}\), are computed by first calculating the unit tangent along the faces:
\[
t_{x,f}=\frac{x_{2}-x_{1}}{A_{f}}
\]

\[
t_{y,f}=\frac{x_{2}-x_{1}}{A_{f}}
\]

Next the normal vector now be computed in 2D:
\[
n_{y,f} = -t_{x,f}
\]

\[
n_{x,f} = t_{y,f}
\]
For 3D, using the tangent vectors \(t_{1}\), \(t_{2}\) (just vector substraction of vertices, not the \textbf{unit} tangent vector):
\[
n = t_{1} \times t_{2}
\]
or 
\[
n=(t_{1y}t_{2z} -t_{1z}t_{2y}) \hat{i} + (t_{1z}t_{2x} -t_{1x}t_{2z}) \hat{j} + (t_{1x}t_{2y} -t_{1y}t_{2x}) \hat{j}
\]
The face areas, \(A_{f}\) in 2D are just the line length given by the Pythagorean theorem. While in 3D, they are given as:
\[
A_{f} = \frac{1}{1} |t_{1} \times t_{2}| = \frac{1}{2} |n|
\]
Lastly, the face centroid components, \(x_{f}\), \(y_{f}\) are given as:
\[
x_{f}=\frac{1}{A_{f}} \int_{S_{f}} x dA
\]

\[
y_{f}=\frac{1}{A_{f}} \int_{S_{f}} y dA
\]

\[
z_{f}=\frac{1}{A_{f}} \int_{S_{f}} z dA
\]



\subsection{Computations:}
\label{sec:org74c94ea}
\subsubsection{a. Triangle given by points: (1,2), (3,5), (-1,1)}
\label{sec:org979675f}

Faces numbering:
\begin{enumerate}
\item (1,2) to (3,5)
\item (3,5) to (-1,1)
\item (-1,1) to (1,2)
\end{enumerate}

\[
 A_{f,1} = \sqrt{(5-2)^{2} + (3-1)^{2}} = \sqrt{13}
 \]

\[
 A_{f,2} = \sqrt{(1-5)^{2} + (-1-3)^{2}} = \sqrt{32}
 \]

\[
 A_{f,3} = \sqrt{(2-1)^{2} + (1+1)^{2}} = \sqrt{5}
 \]

\[
 x_{f,1} = \frac{1+3}{2}=2, \  y_{f,1} = \frac{2+5}{2}=3.5 
 \]
\[
 x_{f,2} = \frac{3-1}{2}=1, \  y_{f,2} = \frac{5+1}{2}=3 
 \]
\[
 x_{f,3} = \frac{-1+1}{2}=0, \  y_{f,3} = \frac{1+2}{2}=1.5 
 \]

\[
 t_{x,f,1} = \frac{3-1}{A_{f,1}} = 0.554 = -n_{y,f,1}, \ t_{y,f,1} = \frac{5-2}{A_{f,1}} = 0.832 = n_{x,f,1}
 \]

\[
 t_{x,f,2} = \frac{-1-3}{A_{f,2}} = -0.707 = -n_{y,f,2} , \ t_{y,f,2} = \frac{1-5}{A_{f,2}} = -0.707 = n_{x,f,2}
 \]

\[
 t_{x,f,3} = \frac{1+1}{A_{f,3}} = 0.894 = -n_{y,f,3}, \ t_{y,f,3} = \frac{2-1}{A_{f,3}} = 0.447 = n_{x,f,3}
 \]

Finally the area of the triangle can be computed:
\[
 V_{0}=\frac{1}{2} \left( 6- 4 +0) + (-7 + 12 -3) \right) = 2
 \]

\subsubsection{b.Tetrahedron given by points: (1,1,1), (1,4,3), (-2,1,4), (4,-2,1)}
\label{sec:orgdb2802e}
Points numbering:
\begin{enumerate}
\item (1,1,1)
\item (1,4,3)
\item (-2,1,4)
\item (4,-2,1)
\end{enumerate}

faces numbering:
\begin{enumerate}
\item includes points: 1,2,3
\item includes points: 2,3,4
\item includes points: 3,4,1
\item includes points: 4,1,2
\end{enumerate}

Tangent vectors can be calculated as the vector difference :
\begin{itemize}
\item Face 1:
\end{itemize}
\[
t_{1,2} = [0,3,2]
\]
\[
t_{1,3} = [-3,0,3]
\]
\[
n_{1} = t_{1,2} \times t_{1,3}= [9,-6,9]
\]
\begin{itemize}
\item Face 2:
\end{itemize}
\[
t_{3,4} = [6,-3,-3]
\]
\[
n_{2} = t_{2,3} \times t_{3,4}= [12,-3,27]
\]
\begin{itemize}
\item Face 3:
\end{itemize}
\[
t_{4,1} = [-3,3,0]
\]

\[
n_{3} = t_{3,4} \times t_{4,1}= [9,9,9]
\]
\begin{itemize}
\item Face 4:
\end{itemize}
\[
n_{4} = t_{4,1} \times t_{1,2}= [6,6,-9]
\]
Using the fact that \(A_{f} = \frac{1}{2} |n|\), and face centroid values:
\[
V_{0}=\frac{1}{3} \left( \sum_{1}^{N_{f,O}} n_{x,f}x_{f}A_{f} + \sum_{1}^{N_{f,O}} n_{y,f}y_{f}A_{f} + \sum_{1}^{N_{f,O}} n_{z,f}z_{f}A_{f} \right) = 7.5
\]

\section{7.2 Calculating average \(\nabla \phi\):}
\label{sec:org9838576}
\end{document}